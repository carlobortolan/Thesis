% !TeX root = ../main.tex
% !TeX root = ../main.tex
% Add the above to each chapter to make compiling the PDF easier in some editors.

\chapter{Conclusion}\label{chapter:conclusion}

Video streaming infrastructure is a fascinating topic often taken for granted by users who open Netflix or YouTube and start streaming videos for hours without an idea of the complexity of what happens behind the screen when they press the "play" button. This thesis has not only shown how enterprise-grade streaming architecture is built and optimized to its limit, but also demonstrated the hands-on process of scaling TUM-Live. Besides scaling it to a distributed system that allows organizations to share computing resources and storage, microservices like the RTMP-Proxy now made features such as self-streaming more accessible. With this, TUM-Live will be able to be used not just by the \ac{CIT}, but also by other \ac{TUM} schools and universities.
Additionally, we have found that there are still areas with room for improvement, including performance bottlenecks, complicated error management, N+1 queries and the issue of having an in-memory chat. Lastly, to provide concrete solutions, optimizations were proposed, such as the \ac{gRPC} \ac{API} that reduced latency by 3x and improved throughput by 8x compared to the current system. 

Hopefully this thesis will serve as a useful resource for developers working on the GoCast architecture, researches exploring new streaming technologies, professionals looking for alternative scaling solutions and all individuals interested in understanding what really happens when they press "play".