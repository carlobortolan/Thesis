% !TeX root = ../main.tex
% !TeX root = ../main.tex
% Add the above to each chapter to make compiling the PDF easier in some editors.

\chapter{Fundamentals of TUM-Live}\label{chapter:fundamentals}

\section{GoCast Lecture Streaming Service}

\subsection{Overview}

GoCast is a fully self-hosted platform for live-streaming and recording of lectures, in use at the \ac{TUM} as \href{https://tum.live}{TUM-Live}.
TUM-Live offers live and on-demand videos of lectures and events from \ac{TUM}'s \href{https://www.cit.tum.de/}{School of Computation and Technology}, mainly by the department of informatics and mathematics. The source code is open-source, accessible at \href{https://github.com/TUM-Dev/gocast}{github.com/TUM-Dev/gocast} and licensed under the MIT license. Its main features include:

\begin{itemize}
    \item Automatic live-streaming from auditoriums based on lecture schedules imported from CAMPUSonline (campus management system used at \ac{TUM} as TUMOnline).
    \item Self-service interface for lecturers to schedule and manage their \ac{VOD}s and streams.
    \item Automated import of lecture schedules and enrollments from CAMPUSonline.
    \item Self-streaming via third party streaming software such as \href{https://obsproject.com/}{OBS}, \href{https://zoom.us}{Zoom}, etc.
    \item Automatic recording of live-streams.
    \item Manual \ac{VOD} uploads.
    \item Automatic post-processing of recordings.
    \begin{itemize}
        \item Detects silence in videos and makes them skip-able.
        \item Transcribes videos and makes them searchable.
        \item Generates thumbnails.
    \end{itemize}
    \item Moderated live chat for listeners to ask questions.
    \begin{itemize}
        \item Polls can be created by lecturers.
        \item Questions can be upvoted by listeners.
        \item Questions can be marked as answered or hidden.
        % \item Optional moderation features for lecturers.
    \end{itemize}
\end{itemize}

\subsection{Current System Architecture}

At the core of the GoCast system, there is the main \ac{API} built on the \href{https://github.com/gin-gonic/gin}{Gin-Gonic} Framework and connected to a \href{https://mariadb.org/}{MariaDB} Database. Its main functionality is to manage users, courses, streams, pull events from CAMPUSonline and schedule tasks. For user authentication, it uses the provided services of \ac{TUM}'s \ac{SSO} and \ac{SAML} to allow users to authenticate themselves with their university credentials. 
Next, whenever a lecture is recorded or livestreamed, the video data is processed by a TUM-Live Worker which then transcodes and segments the video into MPEG-2 compressed video transport stream files. These segments are then copied to a shared storage using the VOD Service component so that they can then later be distributed by the Edge Server.

\begin{figure}[htpb]
    \centering
    \includegraphics[width=420pt]{images/OldDeploymentDiagram2.png}
    \caption[Subsystem Decomposition]{Subsystem Decomposition Model of TUM-Live}\label{fig:old-system-architecture}
\end{figure}

\subsection{User Statistics of TUM-Live}\label{subsection:user-stats-tumlive}

Since its creation in February of 2021, GoCast has been used to stream thousands of hours of video every semester for more than 1,300 courses, 20,000 streams and 30,000 students. The following plots (see \autoref{fig:tumlive-stats}) display a broad overview of viewer metrics from the current system. As the \textit{VoD activity throughout the day} plot shows, the hours at which the users watch recorded \ac{VOD}s is normally distributed, with the mean being around 4PM. Most students use TUM-Live throughout the entire lecture week (see \textit{VoD activity per day of week} plot showing an evenly distributed \ac{VOD} activity over the week), meaning that the Edge Servers need to fully functional at all times. At its peak, there are nearly 6,000 \ac{VOD} replays per day (see \textit{VoD activity per day}) - while at times - mostly during the semester breaks - there are weeks with nearly no \ac{VOD} activity at all.

\begin{figure}[htpb]
    \centering
    \includegraphics[width=\linewidth]{images/TUMLiveStats.png}
    \caption[TUM-Live Statistics]{TUM-Live Viewing Statistics}\label{fig:tumlive-stats}
\end{figure}

What's especially interesting is the development of the number of users, courses and streams over time. As can be seen in \autoref{fig:tumlive-stats-2}, at the beginning of each semester there is a clear spike in new users, created courses and streams. This is mainly due to the automatic import of lecture data and enrollments from CAMPUSonline. Between semesters, the increase in new users and courses is rather flat, with a slight increase in the number of streams (mostly streams that have been created manually).  

\begin{figure}[htpb]
    \centering
    \includegraphics[width=\linewidth]{images/TUMLiveStats2.png}
    \caption[Cumulative Number of Users, Courses and Streams over Time (Log Scale)]{Cumulative Number of Users, Courses and Streams over Time (Log Scale)}\label{fig:tumlive-stats-2}
\end{figure}
