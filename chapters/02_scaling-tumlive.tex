% !TeX root = ../main.tex
% !TeX root = ../main.tex
% Add the above to each chapter to make compiling the PDF easier in some editors.

\chapter{Scaling TUMLive}\label{chapter:introduction}

\section{Introduction of Schools}

Initially, GoCast was used primarily by the former faculty of Informatics at TUM. However, with increasing demand, GoCast needs to be extended for university-wide lecture streaming. The solution to this are schools (not to be confused with TUMOnline's 'TUM School').

To start using GoCast for your department/school, you only need to deploy the TUM-Live Worker, VoD Service and TUM-Live Edge yourself. All other services are already provided by the GoCast network.

The idea is the following: To avoid one entity having to manage and process all streaming data for the entire university (or multiple universities), GoCast is distributed to multiple entities. Each entity (aka GoCast 'school') has so-called maintainers (users with the maintainer user role) that are allowed to manage the school's resources such as Workers/Runners, VoD Services, etc.

Maintainers also have some basic administrative functionality which is limited to their schools' scope (e.g., create, update and delete courses and streams only for those schools which are administered by that maintainer). For an overview of your administered schools, go to the "schools"-tab in the admin dashboard.
info

One maintainer can maintain multiple schools.
The following school-related actions are allowed by a maintainer of a school:

    Create, update or delete school

    Create new tokens for that school (required to add new resources)

    Manage school's resources

    Manage school's maintainers

TUMOnline School vs. GoCast School

TUMOnline has a strict hierarchical structure for its organizations (one school has multiple departments; one department has multiple chairs; one chair has multiple courses ...).
note

On a side node, TUMOnline has 7 schools, 29 departments and 487 chairs.

While GoCast is mainly used by the TUM, in principle it doesn't need to differentiate between organizational types that strictly. Organizations are only relevant when it comes to distributing the livestreams and recordings of a certain entity to that entity's resources (e.g., Workers/Runners and VoD Services). Hence, the introduction of GoCast's "schools" which represent an entity responsible for processing data. In practice, this is most of the time a TUMOnline school, however, in theory one could also create a GoCast "school" for a department, chair or smaller organization, depending on the specific situation.
Here's an example to illustrate this in a more detailled way:

    The TUMOnline "School of Management" (SOM) wants to start using GoCast. Hence, the SOM's IT team contacts the admins of GoCast who then create a new GoCast "school" of type TUM School and assign the SOM IT team as maintainers.

    The subordinated "Chair of Financial Management and Capital Markets" (FA), however, has its own data center and wants to host its lectures with its own resources. In this case, either one of the SOM maintainers or the RBG can create a new GoCast "school" of type Lehrstuhl and accordingly assign new maintainers from the FA-team. Now, the FA-team can connect their own resources from their data center with GoCast, independently of the SOM.

\section{Distributed Resources}



\subsection{Workers}
\subsection{Runners}
\subsection{Ingest Servers}
\subsection{Voice Service}

\section{Shared Resources}
