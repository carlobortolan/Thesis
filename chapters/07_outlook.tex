% !TeX root = ../main.tex
% !TeX root = ../main.tex
% Add the above to each chapter to make compiling the PDF easier in some editors.

\chapter{Outlook and Alternative Technologies}\label{chapter:outlook}

In the previous sections, we looked at possible optimization approaches and limitations of TUM-Live. This chapter goes one step further and takes a brief look at current developments, trends and future technologies in the field of video streaming.

\section{Machine Learning for Cloud Video Streaming Services}

One promising application is predicting the transcoding time of video segments before scheduling, given an input video stream and its transcoding parameters, using the transcoding time as a random variable based on past observations. Simulation results show that such prediction methods can lead to significantly better load balancing of transcoding jobs than traditional methods~\parencite{cloud_predicting}.
This would be especially useful for cloud environments, as accurate task execution time estimations can lower overall usage costs and resource provisioning. Other possible application fields of machine learning are for cost and storage optimization of deployed streaming systems, which could lead up to 15\% to 20\% compared to current methods~\parencite{deep_learning_cloud}.

\section{Blockchain Technology for Video Streaming}

Blockchain is a technology that allows secure \ac{P2P} communication through a shared database called a distributed ledger. In this system, every computer (node) in the network has a copy of the blockchain, which contains the full history of all transactions. Before a transaction can be added, all nodes must agree on it.
Blockchain could potentially impact the video streaming industry, as it could be used to create an allowlist of approved publishers or distributors to control user identities in a decentralized way.
Currently, the video streaming industry is mainly controlled by quality standards and algorithms set by streaming platforms. For example, on YouTube, the search and recommendation systems are controlled by YouTube, not the creator. A Blockchain-supported system changes this by allowing publishers to share their content~\parencite{cloud_streaming}.

\section{Media Over QUIC}

Media over QUIC is built on top of the QUIC protocol\footnote{QUIC (Quick UDP Internet Connections), published in 2013, is a transport protocol designed to address the performance and security limitations of traditional protocols like TCP. It is built on top of UDP and combines features from protocols such as TLS and HTTP/2 to provide low-latency and secure data transmissions~\parencite{quic}.}, focusing on live video delivery and taking advantage of the same benefits that QUIC offers over regular TCP. Like QUIC, Media over QUIC provides reduced connection setup times, encryption by default and multiplexing to avoid head-of-line blocking and congestion control, leading to significantly reduced latency compared to other standards~\parencite{moq_ieft}. It uses one unified protocol for transmitting and receiving media, including audio, video and time-synchronized metadata like captions. Internally, Media over QUIC works by allowing \textit{"media to flow through multiple relays, which can help fan out the data to many downstream users [...] [and] can be placed throughout the data transportation route, including at the \ac{CDN} level, within a 5G network, and even on a user’s local Wi-Fi. As media travels, copies of the data are stored in these relays”}~\parencite{moq_ieft}. Therefore, lost media packets don’t need to be re-requested and re-sent from and to the original server but only from and to the next relay (e.g., the user’s local Wi-Fi).

Currently, Media over QUIC is still in the drafting stage, but there already are early available implementations and it can be expected that once it gets standardized and becomes more widely spread, many streaming services will incorporate it into their platform’s infrastructure~\parencite{moq_project}.

\section{Environmental Considerations}

As explained in previous sections, video streaming services are resource-intensive, performing computing- and storage-intensive tasks that consume large amounts of computational power and storage.
Given the environmental focus nowadays, it might require companies to consider their ecological footprint first. Advancements in video compression standards can reduce overall bandwidth usage but also shift environmental responsibilities from data centers and streaming providers to end users and hardware manufacturers, as more efficient video compression algorithms require additional power to encode and decode video, which increases the energy consumption of end-user devices~\parencite{save_environment}.