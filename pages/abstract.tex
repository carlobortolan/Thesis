\chapter{\abstractname}

Video streaming platforms are used on a daily basis by millions of users and need to be able to provide high-quality videos without interruptions. TUM-Live, used daily by more than 15.000 students at TUM is a perfect example for such a system that consists of multiple services that need to perform reliably with as little human intervention as possible.
In my Bachelor Thesis, I researched different approaches to scaling such a system and increasing stability by extending the current TUM-Live infrastructure to a distributed network managed my external schools, optimizing the TUM-Live API, introducing external resource-management and comparing the current REST API with a prototype for a gRPC API. The goal was to understand, analyze and improve the current infrastructure, as well as assess its potential to scale and identify any potential future limitations such as performance bottlenecks within the Worker subsystem or the problem of having an in-memory chat system.

TODO: Update

How can the existing infrastructure behind video streaming services such as TUM-Live be optimized and scaled while maintaining reliability, stability and high streaming quality?
